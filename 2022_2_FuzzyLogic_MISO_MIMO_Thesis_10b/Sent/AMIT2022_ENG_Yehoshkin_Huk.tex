\documentclass[a5paper, 10pt]{article}

\usepackage[T2A]{fontenc}
\usepackage[cp1251]{inputenc}
\usepackage[english,russian,ukrainian]{babel}

\usepackage{amsmath,amssymb,latexsym,enumerate,AMIT2022}

\begin{document}

\title{The advantages of using MISO and MIMO models in fuzzy expert systems}

\authors{Danylo Yehoshkin, Natalia Huk}

\email{KnightDanila@i.ua, natalyguk29@gmail.com}

\address{Oles Honchar Dnipro National University}

\par
Nowadays, large data sets have been accumulated in various branches of human activity. Analyzing and making decisions based on this data requires experts who can make the right conclusions.
To automate the decision-making process, it is customary to use artificial intelligence systems. The following types of AI are used to solve problems of intelligent systems: expert systems, fuzzy logic, neural networks, genetic algorithms, and others.
An expert system based on fuzzy rules uses fuzzy linguistic variables to express the level of quality assessment and expert criteria. Such fuzzy rules and fuzzy sets are well suited for processing incomplete and formalized data, and their application in knowledge search processes is really useful in terms of human interpretability.
\par
The article \cite{surname_l_1} presents the design and verification of two expert systems for level control in vertical two-tank systems. The structure of a proportional integral fuzzy controller with multiple inputs and multiple outputs is considered. The Mamdani inference mechanism and experimental approximation based on the least squares method for non-linear data were used. The paper  \cite{surname_l_2} considers the solution of problems with a large number of input data and a large number of output data. Also, the input data is highly coherent and not separable without specific information. In such a case, two different modeling strategies can be used, with the option of creating either a single MIMO model or a set of MISO models. 
\par
Afterwards, the input data can be multivariate data set with different dimensions depending on the task, and the expected output of the expert may have a dimension higher than one. Therefore, the fuzzy expert system can take a data set as multivariate input. Afterwards the return output of the system should consist of one consiquent or multivariate consiquent (output dimension). Such systems are called MISO and MIMO. The term MISO itself means �Multiple-Input and Single-Output�, and the term MIMO means �Multiple-Input and Multiple-Output�  \cite{surname_l_3}\cite{surname_l_4}.
\par
The decomposition of the MIMO system into MISO systems is used to speed up the verification process for MIMO rule systems on completeness and correctness,
Also, decomposition allows us to consider the investigated, complex rules system with multiple outputs, as a system consisting of separate, interconnected, simple subsystems. In the general theory of expert systems, it has been proven that most systems can be decomposed using three basic representations of subsystems  \cite{surname_l_5}. Basic representations of subsystems are: cascade, parallel connection of elements, as well as connection by feedback with closure. Systems created by combining multiple representations are called hierarchical fuzzy systems. They also allow the rule to be decomposed into sub-levels for sequential application.
\par
The main task of this article is to automatically generate a full knowledge base consisting of production rules. And reduce rules by different levels of quality. According to the fuzzy approximation theorem \cite{surname_l_6}, any arbitrarily  complex  mathematical dependence can be approximated by a system based on fuzzy logic. The automatically generated knowledge base consists of terms combination. And presented as an Antecedent matrix. Meanwhile, the rule Consequences are presented as a vector. The consequence vector was built by training data set. Evaluating the quality of the model, it is necessary to compare the conclusions obtained using the model with training data or with the results of a real process (object) using the following metrics: accuracy, precision, recall, f1-score \cite{surname_l_7}.
Afterward, the algorithm uses rules with a lower quality. If a satisfactory output is not received, the system switches to rules with higher quality, this allows to speed up the decision-making process, in addition, the rules calculated at the previous stage do not need to be recalculated.
The development of hierarchical expert systems for problems is an important task and allows you to automate the decision-making process in problems with multivariate data and processes.

\begin{thebibliography}{99}
\bibitem{surname_l_1}Claudia-Adina Bojan-Dragos, Elena-Lorena Hedrea, Radu-Emil Precup, Alexandra-Iulia Szedlak-Stinean and Raul-Cristian Roman �MIMO Fuzzy Control Solutions for the Level Control of Vertical Two Tank Systems� // Department of Automation and Applied Informatics, Politehnica University Timisoara, Bd. V. Parvan 2, Timisoara, Romania. -- 2019. -- Pp. 810-817.

\bibitem{surname_l_2}G. Castellano, C. Castiello, A.M. Fanelli, C. Mencar �Knowledge discovery by a neuro-fuzzy modeling framework� / Computer Science Department, University of Bari, Via Orabona, 4, Bari 70126, Italy.  -- 2005. -- Pp. 187-207.

\bibitem{surname_l_3}S. Askari �A novel and fast MIMO fuzzy inference system based on a class of fuzzy clustering algorithms with interpretability and complexity analysis� / Expert Systems with Applications: An International Journal. Volume 84. Issue C. October -- 2017. -- Pp 301-322. https://doi.org/10.1016/j.eswa.2017.04.045

\bibitem{surname_l_4}Amosov O.S., Amosova L.N. Optimal estimation by using fuzzy systems // Proc. of the 17-th World Congress IFAC. -- Seoul, Korea. -- 2008. -- Pp.6094-6099. 

\bibitem{surname_l_5}D. Mesarovic and Yasuhiko Takahara. �General systems theory: mathematical foundations� / Mathematics in science and engineering Volume 113, academic press New York, San Francisco, London. -- Pp 316.

\bibitem{surname_l_6}Kosko B. Fuzzy systems as universal approximators. IEEE Trans. on Computers. -- 1994. 43, N 11. -- Pp 1329?1333. DOI: 10.1109/12.324566.

\bibitem{surname_l_7}Eli Stevens, Deep Learning with PyTorch / Eli Stevens, Luca Antiga, Thomas Viehmann / Manning Publications Co. Shelter Island, NY, -- 2020. -- Pp. 520.
\end{thebibliography}


\end{document} 